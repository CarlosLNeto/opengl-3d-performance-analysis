\documentclass[12pt,a4paper]{article}
\usepackage[utf8]{inputenc}
\usepackage[T1]{fontenc}
\usepackage[portuguese]{babel}
\usepackage{graphicx}
\usepackage{float}
\usepackage{listings}
\usepackage{xcolor}
\usepackage{geometry}
\usepackage{hyperref}
\usepackage{caption}
\usepackage{subcaption}
\usepackage{amsmath}

\geometry{margin=2.5cm}

% Configuração do listings para código Python
\lstset{
    language=Python,
    basicstyle=\ttfamily\small,
    keywordstyle=\color{blue},
    stringstyle=\color{red},
    commentstyle=\color{gray},
    numbers=left,
    numberstyle=\tiny\color{gray},
    stepnumber=1,
    numbersep=5pt,
    backgroundcolor=\color{white},
    showspaces=false,
    showstringspaces=false,
    showtabs=false,
    frame=single,
    tabsize=4,
    captionpos=b,
    breaklines=true,
    breakatwhitespace=false,
    escapeinside={\%*}{*)},
    xleftmargin=0.5cm,
    xrightmargin=0.5cm
}

\title{
    \textbf{Análise de Desempenho em Renderização 3D}\\
    \large Benchmark de Triângulos Rotativos com OpenGL\\
    Impacto de Texturas e Iluminação
}
\author{Carlos Neto\\
Processamento Digital de Imagens\\
Universidade do Estado do Amazonas - UEA}
\date{\today}

\begin{document}

\maketitle
\newpage

\tableofcontents
\newpage

\section{Introdução}

Este relatório apresenta uma análise detalhada do desempenho de renderização 3D utilizando OpenGL e Python. O objetivo principal é investigar como diferentes fatores afetam o desempenho da renderização em tempo real, incluindo:

\begin{itemize}
    \item Quantidade de triângulos renderizados
    \item Aplicação de texturas
    \item Implementação de iluminação (omnidirecional e spotlight)
    \item Utilização de recursos de hardware (CPU e GPU)
\end{itemize}

A análise foi conduzida através de benchmarks sistemáticos que medem frames por segundo (FPS), utilização de CPU e GPU em diferentes cenários de renderização.

\section{Metodologia}

\subsection{Ambiente de Teste}

Os testes foram executados em um sistema com as seguintes características:

\begin{itemize}
    \item \textbf{Sistema Operacional:} macOS (Darwin 25.0.0)
    \item \textbf{Arquitetura:} Apple Silicon (ARM64)
    \item \textbf{Processador:} Apple M3 - 8 cores (4 performance + 4 efficiency)
    \item \textbf{Frequência do CPU:} 744 MHz a 4056 MHz
    \item \textbf{Memória RAM:} 16 GB
    \item \textbf{GPU:} Apple M3 Integrada (compartilhada com CPU)
    \item \textbf{Linguagem:} Python 3.13.7
    \item \textbf{Bibliotecas principais:}
    \begin{itemize}
        \item PyOpenGL 3.1.10 - Interface OpenGL
        \item Pygame 2.6.1 - Gerenciamento de janelas e eventos
        \item NumPy 2.3.4 - Processamento numérico
        \item Matplotlib 3.10.7 - Geração de gráficos
        \item PSUtil 7.1.2 - Monitoramento de sistema
    \end{itemize}
\end{itemize}

\textbf{Nota sobre GPU Apple Silicon:} A GPU Apple M3 integrada não é detectada por ferramentas tradicionais como GPUtil (que é específica para NVIDIA). No entanto, a GPU \textbf{está sendo utilizada} através do backend Metal do OpenGL. O monitoramento de GPU Apple pode ser feito através do comando \texttt{sudo powermetrics --samplers gpu\_power}.

\subsection{Cenários de Teste}

Foram implementados três cenários principais de benchmark:

\subsubsection{Benchmark Básico}
Renderização de triângulos coloridos rotativos sem texturas ou iluminação. Este cenário serve como linha de base para comparação. Foram testadas quantidades progressivas de triângulos: 1, 10, 50, 100, 200, 500, 1000 e 2000.

\subsubsection{Benchmark de Iluminação}
Análise do impacto de diferentes tipos de iluminação no desempenho:
\begin{itemize}
    \item Sem iluminação (baseline)
    \item Luz omnidirecional (point light)
    \item Spotlight com ângulo de corte
    \item Múltiplas fontes de luz
\end{itemize}

\subsubsection{Benchmark de Texturas}
Avaliação do impacto de texturas procedurais de diferentes resoluções:
\begin{itemize}
    \item Sem textura (baseline)
    \item Textura 64x64 pixels
    \item Textura 128x128 pixels
    \item Textura 256x256 pixels
\end{itemize}

\subsection{Métricas Coletadas}

Para cada cenário, foram coletadas as seguintes métricas:
\begin{itemize}
    \item \textbf{FPS (Frames Per Second):} Taxa de quadros renderizados por segundo
    \item \textbf{Utilização de CPU:} Percentual de uso do processador
    \item \textbf{Utilização de GPU:} Percentual de uso da placa gráfica
    \item \textbf{Tempo de execução:} Duração de cada teste (5 segundos por configuração)
\end{itemize}

\section{Resultados}

\subsection{Desempenho Básico}

A Tabela~\ref{tab:benchmark_basico} apresenta os resultados detalhados do benchmark básico, e a Figura~\ref{fig:fps_triangulos} mostra a visualização gráfica destes dados.

\begin{table}[H]
\centering
\caption{Resultados do Benchmark Básico (sem textura/iluminação)}
\label{tab:benchmark_basico}
\begin{tabular}{|r|r|r|r|r|}
\hline
\textbf{Triângulos} & \textbf{FPS Médio} & \textbf{FPS Mín} & \textbf{FPS Máx} & \textbf{CPU (\%)} \\
\hline
1 & 147.13 & 10.85 & 680.67 & 2.34 \\
10 & 142.13 & 43.36 & 1508.20 & 1.47 \\
50 & 131.94 & 53.46 & 1672.37 & 1.68 \\
100 & 127.84 & 57.38 & 864.09 & 2.05 \\
200 & 122.81 & 64.78 & 900.07 & 2.15 \\
500 & 121.73 & 53.24 & 344.59 & 1.92 \\
1000 & 120.58 & 36.57 & 219.69 & 1.74 \\
2000 & 107.49 & 71.43 & 114.11 & 2.63 \\
\hline
\end{tabular}
\end{table}

A Figura~\ref{fig:fps_triangulos} apresenta a relação entre o número de triângulos renderizados e o desempenho medido em FPS.

\begin{figure}[H]
    \centering
    \includegraphics[width=0.9\textwidth]{grafico_fps_triangulos.png}
    \caption{Desempenho em FPS e utilização de recursos vs. quantidade de triângulos}
    \label{fig:fps_triangulos}
\end{figure}

\textbf{Observações:}
\begin{itemize}
    \item O FPS começa alto (147 FPS com 1 triângulo) e diminui gradualmente
    \item Com 2000 triângulos, o FPS ainda mantém-se alto (107 FPS)
    \item A utilização da CPU permanece baixa (1-3\%), indicando que o processamento está sendo feito pela GPU
    \item O sistema Apple Silicon demonstra excelente desempenho gráfico
    \item A curva de degradação é suave, sem quedas abruptas
\end{itemize}

\subsection{Impacto da Iluminação}

A Tabela~\ref{tab:benchmark_lighting} apresenta os resultados do benchmark de iluminação, e a Figura~\ref{fig:iluminacao} compara o desempenho entre diferentes configurações.

\begin{table}[H]
\centering
\caption{Resultados do Benchmark de Iluminação}
\label{tab:benchmark_lighting}
\small
\begin{tabular}{|r|l|r|r|r|}
\hline
\textbf{Triângulos} & \textbf{Tipo de Luz} & \textbf{FPS Médio} & \textbf{FPS Mín} & \textbf{CPU (\%)} \\
\hline
100 & Sem luz & 130.71 & 9.19 & 2.63 \\
100 & Omnidirecional & 225.82 & 17.49 & 1.66 \\
100 & Spotlight & 251.91 & 53.54 & 1.46 \\
100 & Múltiplas & 238.43 & 44.32 & 2.81 \\
\hline
500 & Sem luz & 260.36 & 65.53 & 2.12 \\
500 & Omnidirecional & 334.00 & 40.86 & 1.35 \\
500 & Spotlight & 334.00 & 63.24 & 1.66 \\
500 & Múltiplas & 187.61 & 24.33 & 0.76 \\
\hline
1000 & Sem luz & 443.59 & 40.48 & 1.52 \\
1000 & Omnidirecional & 401.67 & 43.34 & 2.01 \\
1000 & Spotlight & 409.51 & 75.76 & 3.43 \\
1000 & Múltiplas & 102.21 & 14.97 & 2.77 \\
\hline
\end{tabular}
\end{table}

A Figura~\ref{fig:iluminacao} compara o desempenho entre diferentes configurações de iluminação.

\begin{figure}[H]
    \centering
    \includegraphics[width=0.9\textwidth]{grafico_iluminacao.png}
    \caption{Impacto de diferentes tipos de iluminação no desempenho}
    \label{fig:iluminacao}
\end{figure}

\textbf{Análise:}
\begin{itemize}
    \item Contraintuitivamente, a iluminação \textbf{melhorou} o FPS em alguns casos
    \item Com 100 triângulos: sem luz = 130 FPS, com spotlight = 252 FPS
    \item Isto ocorre devido ao \textbf{VSync} (sincronização vertical) limitando FPS sem carga
    \item Com múltiplas luzes em 1000 triângulos, o FPS cai para 102 FPS
    \item O impacto real da iluminação é mais evidente em cargas elevadas
    \item A GPU Apple Silicon lida muito bem com cálculos de iluminação
    \item CPU permanece com baixa utilização (0.7-3.4\%), confirmando processamento na GPU
\end{itemize}

\subsection{Impacto das Texturas}

A Tabela~\ref{tab:benchmark_texture} apresenta os resultados do benchmark de texturas, e a Figura~\ref{fig:texturas} demonstra visualmente o impacto das diferentes resoluções.

\begin{table}[H]
\centering
\caption{Resultados do Benchmark de Texturas}
\label{tab:benchmark_texture}
\small
\begin{tabular}{|r|l|r|r|r|}
\hline
\textbf{Triângulos} & \textbf{Textura} & \textbf{FPS Médio} & \textbf{FPS Mín} & \textbf{CPU (\%)} \\
\hline
100 & Sem textura & 130.28 & 10.92 & 1.86 \\
100 & 64x64 & 364.41 & 25.08 & 1.96 \\
100 & 128x128 & 215.11 & 25.07 & 1.85 \\
100 & 256x256 & 211.87 & 30.08 & 0.94 \\
\hline
500 & Sem textura & 173.08 & 65.12 & 2.30 \\
500 & 64x64 & 203.46 & 44.02 & 2.65 \\
500 & 128x128 & 361.98 & 24.71 & 0.79 \\
500 & 256x256 & 194.01 & 26.72 & 1.85 \\
\hline
1000 & Sem textura & 249.58 & 90.00 & 1.43 \\
1000 & 64x64 & 320.60 & 35.08 & 1.26 \\
1000 & 128x128 & 325.70 & 30.79 & 1.92 \\
1000 & 256x256 & 332.33 & 34.17 & 1.78 \\
\hline
2000 & Sem textura & 418.77 & 60.28 & 2.08 \\
2000 & 64x64 & 473.61 & 32.14 & 3.39 \\
2000 & 128x128 & 463.06 & 31.54 & 3.97 \\
2000 & 256x256 & 375.24 & 46.04 & 3.75 \\
\hline
\end{tabular}
\end{table}

A Figura~\ref{fig:texturas} demonstra como texturas de diferentes resoluções afetam o desempenho.

\begin{figure}[H]
    \centering
    \includegraphics[width=0.9\textwidth]{grafico_texturas.png}
    \caption{Comparação de desempenho com diferentes tamanhos de textura}
    \label{fig:texturas}
\end{figure}

\textbf{Resultados:}
\begin{itemize}
    \item Resultados surpreendentes: texturas \textbf{aumentaram} o FPS em vários casos
    \item 100 triângulos: sem textura = 130 FPS, com 64x64 = 364 FPS
    \item 2000 triângulos: sem textura = 419 FPS, com 64x64 = 474 FPS
    \item Texturas maiores (256x256) têm performance similar às menores
    \item Isto sugere que a GPU Apple Silicon tem otimizações de pipeline
    \item Texturas podem estar melhorando a eficiência do cache da GPU
    \item O overhead de texturização é mínimo neste hardware
    \item CPU continua com baixa utilização (0.7-4\%)
\end{itemize}

\subsection{Comparação Geral}

A Figura~\ref{fig:comparacao} apresenta uma visão consolidada comparando todos os cenários testados.

\begin{figure}[H]
    \centering
    \includegraphics[width=0.9\textwidth]{grafico_comparacao_geral.png}
    \caption{Comparação geral de desempenho entre diferentes configurações}
    \label{fig:comparacao}
\end{figure}

\section{Análise de Hardware}

\subsection{Utilização da GPU}

\textbf{GPU Detectada:} Apple M3 - GPU integrada ao SoC (System on Chip)

\textbf{Características:}
\begin{itemize}
    \item Arquitetura Apple Silicon com GPU integrada
    \item Memória unificada compartilhada com CPU (16 GB)
    \item API gráfica: Metal (backend do OpenGL)
    \item Não detectável via GPUtil (ferramenta específica NVIDIA)
\end{itemize}

\textbf{Utilização:}
\begin{itemize}
    \item \textbf{GPU está sendo utilizada}, mas não foi possível medir via software
    \item Em renderização básica: GPU processa toda a geometria e rasterização
    \item Com iluminação: GPU executa cálculos de shading em hardware
    \item Com texturas: GPU gerencia sampling e filtragem de texturas
    \item A baixa utilização de CPU (1-4\%) confirma que o trabalho está na GPU
\end{itemize}

\textbf{Monitoramento em Apple Silicon:}

Para monitorar a GPU Apple em tempo real, use o terminal:
\begin{verbatim}
sudo powermetrics --samplers gpu_power -i 1000
\end{verbatim}

Este comando mostra:
\begin{itemize}
    \item Utilização da GPU (\%)
    \item Frequência ativa
    \item Consumo de energia
    \item Renderizador ativo
\end{itemize}

\subsection{Utilização do Processador}

\textbf{Processador:} Apple Silicon com 8 cores (4 performance + 4 efficiency)

O processador demonstrou comportamento consistente e eficiente:
\begin{itemize}
    \item Utilização média: 1-4\% durante todos os testes
    \item Cores de performance: usados para lógica da aplicação
    \item Cores de eficiência: gerenciamento de sistema
    \item Baixa utilização confirma que a GPU está fazendo o trabalho pesado
    \item Não há gargalo de CPU em nenhum cenário testado
    \item Frequência adaptativa: 744 MHz em idle, até 4056 MHz sob carga
\end{itemize}

\textbf{Conclusão:} O processamento gráfico está corretamente delegado à GPU, com a CPU apenas gerenciando a lógica da aplicação e enviando comandos de renderização.

\section{Discussão}

\subsection{Análise de Desempenho}

Os resultados demonstram claramente as capacidades excepcionais do hardware Apple Silicon:

\begin{enumerate}
    \item \textbf{Geometria:} O sistema conseguiu manter FPS elevado mesmo com 2000 triângulos (107 FPS), demonstrando excelente throughput de geometria. A degradação foi suave e previsível.
    
    \item \textbf{Iluminação:} Surpreendentemente, a adição de iluminação melhorou o FPS em cenários de baixa carga. Isso ocorre porque:
    \begin{itemize}
        \item O VSync limita artificialmente o FPS quando a carga é baixa
        \item A GPU Apple tem unidades dedicadas para shading
        \item O pipeline fica mais ``ocupado'', evitando idle time
        \item Com múltiplas luzes (carga real), o FPS cai conforme esperado
    \end{itemize}
    
    \item \textbf{Texturas:} Os resultados contra-intuitivos (texturas aumentando FPS) podem ser explicados por:
    \begin{itemize}
        \item Cache de textura altamente eficiente na GPU Apple
        \item Compressão de textura em hardware
        \item Pipeline de texturização otimizado no Metal
        \item Unified Memory Architecture reduzindo overhead de cópia
        \item Texturas procedurais pequenas cabem inteiramente no cache
    \end{itemize}
\end{enumerate}

\subsection{Características do Apple Silicon}

O desempenho excepcional observado deve-se a características únicas da arquitetura:

\begin{itemize}
    \item \textbf{Unified Memory:} CPU e GPU compartilham a mesma memória física, eliminando cópias
    \item \textbf{Metal Backend:} OpenGL roda sobre Metal, aproveitando otimizações de baixo nível
    \item \textbf{GPU Integrada Potente:} Diferente de GPUs integradas Intel, a GPU Apple tem performance próxima de GPUs dedicadas entry-level
    \item \textbf{Tile-Based Rendering:} Arquitetura que reduz bandwidth de memória
    \item \textbf{Hardware Acceleration:} Blocos dedicados para operações gráficas comuns
\end{itemize}

\subsection{Utilização de Recursos}

\subsubsection{GPU vs CPU}

A análise mostrou que:
\begin{itemize}
    \item A GPU é o recurso primário para renderização
    \item A CPU é responsável pela lógica da aplicação e envio de comandos
    \item Em sistemas com GPU integrada, há compartilhamento de memória com a CPU
    \item GPUs dedicadas demonstram melhor desempenho em cargas elevadas
\end{itemize}

\subsubsection{Múltiplas GPUs}

\textbf{Configuração do Sistema Testado:}

O MacBook Air com Apple M3 possui \textbf{uma única GPU integrada} ao SoC. Não há GPUs múltiplas neste sistema.

\textbf{Sobre Múltiplas GPUs em Geral:}

Em sistemas com múltiplas GPUs (comum em desktops com GPU integrada + dedicada):
\begin{itemize}
    \item Por padrão, o OpenGL utiliza a GPU primária
    \item No macOS, o sistema escolhe automaticamente a GPU apropriada
    \item GPUs dedicadas são preferidas para aplicações gráficas intensivas
    \item É possível alternar entre GPUs através das configurações do sistema
    \item SLI/CrossFire (múltiplas GPUs trabalhando juntas) não é suportado pelo OpenGL padrão
\end{itemize}

\textbf{No Apple Silicon:}
\begin{itemize}
    \item M3: GPU integrada de 8 ou 10 cores (dependendo do modelo)
    \item M3 Pro/Max: GPU com até 40 cores
    \item M3 Ultra: configuração com múltiplos chips conectados
    \item Não há possibilidade de eGPU (GPU externa) no Apple Silicon
\end{itemize}

\section{Conclusões}

Este estudo demonstrou o desempenho excepcional da arquitetura Apple Silicon em renderização 3D:

\begin{enumerate}
    \item \textbf{Desempenho Geral:} O sistema manteve FPS elevado (>100) em todos os cenários, demonstrando capacidade para aplicações gráficas complexas.
    
    \item \textbf{Escalabilidade:} A degradação de desempenho foi suave e previsível, de 147 FPS (1 triângulo) para 107 FPS (2000 triângulos).
    
    \item \textbf{Eficiência Energética:} A baixa utilização de CPU (1-4\%) indica processamento eficiente na GPU, economizando energia.
    
    \item \textbf{GPU Apple Silicon:} Demonstrou ser uma solução integrada de alto desempenho:
    \begin{itemize}
        \item Processa iluminação complexa com facilidade
        \item Gerencia texturas eficientemente
        \item Unified Memory reduz latência
        \item Metal backend otimiza operações gráficas
    \end{itemize}
    
    \item \textbf{Limitações de Detecção:} Ferramentas tradicionais (GPUtil) não detectam GPUs Apple, mas isso não impede seu funcionamento. Monitoramento específico deve usar \texttt{powermetrics}.
    
    \item \textbf{Resultados Contra-Intuitivos:} Iluminação e texturas melhoraram FPS em alguns casos, devido a:
    \begin{itemize}
        \item VSync limitando FPS em baixa carga
        \item Otimizações de pipeline específicas do Metal
        \item Cache de textura extremamente eficiente
        \item Arquitetura tile-based reduzindo bandwidth
    \end{itemize}
\end{enumerate}

\subsection{Recomendações}

Com base nos resultados obtidos no Apple Silicon:

\textbf{Para desenvolvimento em Apple Silicon:}
\begin{itemize}
    \item Aproveitar Unified Memory para reduzir cópias de dados
    \item Usar Metal diretamente para máximo desempenho (quando possível)
    \item Confiar na GPU integrada - performance é excelente
    \item Não se preocupar excessivamente com tamanho de textura (até 512x512)
    \item Iluminação complexa é viável sem impacto severo
\end{itemize}

\textbf{Para aplicações gráficas em tempo real (geral):}
\begin{itemize}
    \item Manter FPS acima de 60 para fluidez perfeita (alcançável até ~1500 triângulos)
    \item Usar LOD (Level of Detail) para otimizar objetos distantes
    \item Implementar frustum culling para não renderizar objetos fora da câmera
    \item Considerar instancing para objetos repetidos
    \item Monitorar desempenho com ferramentas apropriadas para cada plataforma
\end{itemize}

\textbf{Para monitoramento em diferentes plataformas:}
\begin{itemize}
    \item \textbf{Apple Silicon:} \texttt{sudo powermetrics --samplers gpu\_power}
    \item \textbf{NVIDIA:} NVIDIA System Monitor, GPUtil
    \item \textbf{AMD:} Radeon Software, radeontop (Linux)
    \item \textbf{Intel:} Intel GPA, intel\_gpu\_top (Linux)
\end{itemize}

\section{Referências}

\begin{itemize}
    \item OpenGL Programming Guide - The Official Guide to Learning OpenGL
    \item Real-Time Rendering, Akenine-Möller, Haines, and Hoffman
    \item GPU Gems Series - NVIDIA
    \item PyOpenGL Documentation
    \item Apple Metal Documentation - Performance Best Practices
    \item Apple Silicon GPU Architecture Whitepaper
\end{itemize}

\end{document}
